\chapter{Conclusion}
\label{chap:conclusion}
\section{Summary}
\label{sec:summary}
In this thesis, a diverse set of existing algorithms for pupil detection, each demonstrating its unique strengths and weaknesses, have been examined. These algorithms were analyzed under various imaging conditions, giving insight into their robustness and efficiency.

The Results show that finding a robust and efficient combination of algorithms is challenging and that the proposed algorithm cannot solve this task in real-time. The proposed algorithm presents a valid combination but still has weaknesses and needs adaptation to the specific use cases. The parameters must be repeatedly tuned on each data set and can not be seen as just working. The thesis aimed to evaluate and compare different algorithms and show their limitations, and this goal was achieved, and an overview of different approaches is given. The proposed algorithm can only be regarded as a solution that outperforms every method in some use cases.
Although the combination of Haar-like features, ACWE and RANSAC can perform well in even tricky conditions, there is still room for improvement and further research. Especially the Haar-like feature detection stood out with exceptional reliability for finding points inside the pupil and is considered the best method to gain information about the pupil's location.

The estimation of the pupil boundary with the proposed algorithm is valid but not perfect. The ellipse fit can still be improved by further tuning the parameters of the individual algorithms. The performance can be increased by implementing the code into C++ and using more multithreading and parallelization. It is unknown if the proposed algorithm can run in real time, but there is a high chance it is possible with further improvements. The estimation quality is sufficient to be used as a starting point for training a highly reliable AI-based eye Tracker or Iris recognition system and gaze tracing.  

The code published on GitHub can be used as a starting point to build more complex tasks on.  