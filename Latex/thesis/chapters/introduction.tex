% Some commands used in this file
\newcommand{\package}{\emph}

\chapter{Introduction}
\label{chap:introduction}
Pupil detection on images is a fundamental task with diverse applications in various fields, for example computer vision, human-computer interaction and biometric systems. Accurate and efficient pupil detection is crucial in tasks such as gaze tracking \cite{arar_robust_2017}, iris recognition \cite{wildes_iris_1997}, emotion recognition \cite{zheng_multimodal_2014} and medical diagnosis \cite{grubisic_natural_2014}.

Over the years, numerous algorithms and techniques have been proposed for pupil detection, each with strengths and limitations. However, most of these algorithms cannot perform well under various conditions. Consequently, finding a robust and efficient combination of algorithms for pupil detection is still an intriguing research problem. 
\section{Problem Statement}
\label{sec:problem_statement}
The problem of pupil detection discussed in this thesis is the optimal combination of algorithms for pupil detection, given the variability in imaging conditions, including variations in illumination, head poses, occlusions and quality of the images.

A single algorithm often fails to perform well under a wide span of conditions and lacks robustness and efficiency. Therefore this thesis aims to find a combination of algorithms that can perform well over a wide range of conditions without introducing machine learning, therefore, is independent of training data. 
\section{Motivation}
\label{sec:motivation}
The motivation behind this research is the lack of robustness and accuracy of current algorithms for pupil detection. Despite the extensive research on pupil detection no universal solution exists that can consistently handle the wide range of imaging conditions encountered in real-world scenarios. Therefore, exploring algorithmic combinations that can perform well under a wide range of conditions is a challenging and exciting research problem. 

Improving the pupil detection algorithm's performance can increase the precision and reliability of medical diagnosis, human-computer interaction and biometric systems. The benefit of exploring different approaches for pupil detection is the possibility of combining the strengths of different algorithms to achieve a more robust and efficient solution.

This thesis presents a comprehensive analysis of various algorithms for pupil detection and proposes a combination of algorithms. Through extensive experimentation and evaluation, the thesis aims to provide valuable insights and guidelines for researchers and practitioners working on pupil detection. Overall, this research seeks to contribute to developing more effective pupil detection systems that can operate reliably under varying imaging conditions, paving the way for improved applications and advancing the understanding of pupil detection. 
