\chapter{Theory}

In this chapter will take a look at commonly used algorithms in image processing for edge detection, identifyig areas of interest and refining them. 
We discuss those algorithms first in theory and then show it's use case in detecting the iris of an human eye. 

To show the nature of the algorithm, the same preprocessed image is used and therefore it is possible to showcase the results and compare the algorithms in a later chapter
\section{Algorithms}




\subsection{Preprocessing}
To be able to compare the different approaches, it is important to define the used algorithms that lead to the wanted result.
An image taken has to be preprocessed first. The idea behind this step is to create a common ground for narrowing the deviation of the images down, so that the algorithms are 
able to recreate the same result over span of different images.

\subsubsection{Histogram matching}
The algorithms are trimmed to a specific Histogram to work the best. Therefore all Images have to be preprocessed. To fullfil this requirement, Histogram matching is done. In the first step
we take a look at the histogram used to match the others onto. 

\section{Gaussian blur}
The idea behind using a Gaussian blur on a Image first is to minimize unnecessary information in to be processed image. By using the gaussian blur the image loses details and 
this increases the chance to find the main edges in the image, finding the outline of the iris. 


Dummy text.

\subsection{Definition RANSAC}

Dummy text.


\section{Canny Edge Detection}

Dummy text.

\subsection{Definition Canny Edge Detection}

Dummy text.

\section{Active Contur}

Dummy text.

\subsection{Definition Active Countur}

Dummy text.

\subsubsection{Example Subsubsection}

Dummy text.

\paragraph{Example Paragraph}

Dummy text.

\subparagraph{Example Subparagraph}

Dummy text.
